\begin{thebibliography}{}

\bibitem{russian-text}
\link{https://edu-support.mephi.ru/materials/230/edu/misc/russian.md?to=html}~---
Рекомендации по русскоязычному набору текста.

\bibitem{convex}
\link{https://edu-support.mephi.ru/materials/214/edu/lectures/13/lecture\_\_2.md?to=html}~---
Описание проекта «Выпуклая оболочка».

\bibitem{ind-functions}
\link{https://edu-support.mephi.ru/materials/214/edu/lectures/09/lect-29.md?to=html}~---
Теория индуктивных функций.

\bibitem{python}
\link{https://www.python.org/}~---
Официальный сайт языка Python.

\bibitem{rlatex}
С.М. Львовский.
{\em Набор и вёрстка в системе \LaTeX, 3-е изд., испр. и доп.}~---
М., МЦНМО, 2003. Доступны исходные тексты этой книги.

\bibitem{texbook}
D.~E.~Knuth. {\em The \TeX{}book.}~---
Addison-Wesley, 1984. Русский перевод:
Дональд~Е.~Кнут.
{\em Все про \TeX.}~--- Протвино, РД\TeX, 1993.

\bibitem{roganov-jurists}
Е.А. Роганов, Н.Б. Тихомиров, А.М. Шелехов.
{\em Математика и информатика для юристов.}~---
М., МГИУ, 2005.
Доступны исходные тексты этой книги.

\end{thebibliography}
