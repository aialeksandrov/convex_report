\section{Введение}

Проект «Выпуклая оболочка»\cite{convex} решает задачу индуктивного 
перевычисления 
выпуклой оболочки последовательно поступающих точек плоскости и таких её
характеристик, как периметр и площадь. Целью данной работы является
определение периметра и площади части выпуклой оболочки, расположенной
в верхней полуплоскости. Решение этой задачи требует знания теории индуктивных
функций~\cite{ind-functions}, основ аналитической геометрии и векторной алгебры
и языка Python~\cite{python}.

Для подготовки пояснительной записки необходимо знакомство с программой
компьютерной вёрстки \LaTeX~\cite{rlatex}, умение набирать математические 
формулы~\cite{texbook} и включать в документ графические изображения и исходные
коды программ на языке Python.

Общее количество строк в модифицированном проекте составляет около $n$,
из которых более $m$ были изменены или добавлены автором в процессе работы
над решением задачи модификации.
